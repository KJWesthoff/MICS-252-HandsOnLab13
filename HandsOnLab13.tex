%%%%%%%%%%%%%%%%%%%%%%%%%%%%%%%hhhhhh%%%%%%%%%
% APA Assignment Article
% LaTeX Template
% Version 2.0 (February 7, 2023)
%
% This template originates from:
% https://www.LaTeXTemplates.com
%
% Author:
% Vel (vel@latextemplates.com)
%
% License:
% CC BY-NC-SA 4.0 (https://creativecommons.org/licenses/by-nc-sa/4.0/)
%
% NOTE: The bibliography needs to be compiled using the biber engine.
%
%%%%%%%%%%%%%%%%%%%%%%%%%%%%%%%%%%%%%%%%%

%----------------------------------------------------------------------------------------
%	PACKAGES AND OTHER DOCUMENT CONFIGURATIONS
%----------------------------------------------------------------------------------------

\documentclass[
	letterpaper, % Paper size, use either a4paper or letterpaper
	10pt, % Default font size, can also use 11pt or 12pt, although this is not recommended
	unnumberedsections, % Comment to enable section numbering
	twoside, % Two side traditional mode where headers and footers change between odd and even pages, comment this option to make them fixed
]{APAAssignment}

\addbibresource{bibliography.bib} % BibLaTeX bibliography file

\runninghead{MICS CYBER 252, Fall-2024 Hands On Lab Unit 13} % A shortened article title to appear in the running head, leave this command empty for no running head

\footertext{\textit{Hands On Lab 13} (MICS CYBER 252, Fall -2024)} % Text to appear in the footer, leave this command empty for no footer text

\setcounter{page}{1} % The page number of the first page, set this to a higher number if the article is to be part of an issue or larger work

%----------------------------------------------------------------------------------------
%	TITLE SECTION
%----------------------------------------------------------------------------------------

\usepackage[title,toc,titletoc]{appendix}
\usepackage{titlesec}
\usepackage{lscape}
\usepackage{fontawesome}

\title{Hands On Lab: Unit 13 \\ MICS-252, Fall 2024 \\ Digital Forensics II \\ Image Analysis}

% Authors are listed in a comma-separated list with superscript numbers indicating affiliations
% \thanks{} is used for any text that should be placed in a footnote on the first page, such as the corresponding author's email, journal acceptance dates, a copyright/license notice, keywords, etc
% Affiliations are output in the \date{} command
\date{UC Berkeley School of Information \\
MICS Course 252 Fall 2024 (Kristy Westphal)
}


\author{
	Prepared by: Karl-Johan Westhoff \\
	email: \href{mailto:kjwesthoff@berkeley.edu}{kjwesthoff@berkeley.edu}
}


% % Full-width abstract
% \renewcommand{\maketitlehookd}{%
% 	\begin{abstract}
% 		\noindent Lorem ipsum dolor sit amet,rta porttitor.
% 	\end{abstract}
% }

%----------------------------------------------------------------------------------------

\setcounter{tocdepth}{5}
\setcounter{secnumdepth}{5}
\usepackage[title]{appendix}

\begin{document}
\onecolumn
\maketitle % Output the title section

%----------------------------------------------------------------------------------------
%	ARTICLE CONTENTS
%----------------------------------------------------------------------------------------

\section{Introduction}
We are given an Encase .E01 image: "Windows Image.E01" with a size of 10.6 GB. The image was ingested into Autopsy version 4.21 (a lengthy process).
of the image.

\subsection{Objectives}
We are given the following objective for the exercise to get thoroughly around the disc image: \\
"Attempt to answer the following questions:"
\begin{itemize}
	\item What time zone is the image set in?
	\item What Operating System is the suspect drive using?
	\item What event happened on January 1, 1980?
	\item What is the IP address of the suspect drive?
	\item What is the EvenMoreSecretStuff.vhd (and can you see what is in it?)
	\item Do any of the suspicious items discovered look suspicious to you?
	\item Anything else that you found that you want to highlight?
\end{itemize}

\subsection{Autopsy report}
Autopsy has functionality to generate a report of the findings in various formats. I have published the report on github pages for reference in this report here:
\begin{itemize}
	\item	\cite{AutoReport}: \url{https://kjwesthoff.github.io/252-Lab13-AutopsyReport/}
\end{itemize}


\section{Image Ingestion}
The file name indicates that the file is a Windows pc image. The image was subjected to the ingest modules from the default Autopsy 4.21 installation (See Case Summary section of \cite{AutoReport}). The execution of various ingestion modules was a lengthy process (>24hrs. in this case). Some of the ingress modules do not run under Linux, these include "Yara" (runs rules checking files for malware) and "Plaso" which extracts information from the windows registry of the image.
Autopsy can include custom ingestion modules that analyse the file hashes against known malware, in this case hashes were generated for all files (hence the lengthy ingest) and are thus ready for comparison against a database of known hashes (considered out of scope for this assignment). Luckily most of Autopsy's functionality is available while it analyzes the files, and various result sections are populated in the GUI as they become available.

\section{Analysis Results}
The Image holds two data sources; Windows Image.E01 and EvenMoreSecretStuff.vhd, the .vhd extension indicates that it is a Windows Virtual Hard Disc \cite{WindowsVHD}
\subsection{Assignment Objectives}

\subsubsection{Time Zone}
Both data sources indicate: "America/Los Angeles", and the images were acquired on Mar. 20, 2019 in the evening around 8pm, see Appendix \ref{app:MetaData}

\subsubsection{Operating System}
Both the .E01 file name, the .vhd part and various artifacts indicate a Windows OS, the Autopsy report \cite{AutoReport} indicated:

\begin{itemize}
	\item OS: "Windows 10 Enterprise"
	\item product ID "00329-00000-00003-AA856"
	\item built for: "AMD64 architecture"
	\item with the business like computer name: "DESKTOP-0QT8017"
\end{itemize}

\subsubsection{Event on January 1, 1980:}
January 1 1980 at midnight is the epoch (beginning of time) for MS-DOS\footnote{I was about to write that it was when "Skynet became self-aware" but that was August 29, 1997 \cite{enwiki:Terminator}}. For unix systems it is January 1. 1970 (of course Microsoft had to have their own epoch). When computers get confused or are missing a time stamp they default to epoch. In this case it is apparently some google chrome files that is causing some confusion, see Appendix \ref{app:Jan1980}

\subsubsection{IP of "Suspect Drive"}
It appears that the machine was on a local network with IP address 10.0.1.5, when searching for IPv4 addresses it shows up second most after 6.0.0.0 (which I think is not an ip address in this context, but associated with version of "Microsoft-Windows-ServicingStack"), see Appendix \ref{app:IpHits}. The ip 10.0.1.5 shows up in files also containing the amd64 CPU architecture.


\subsubsection{EvenMoreSecretStuff.vhd}
The .vhd extension suggests it is a "Virtual Hard Disc", the format is related to Microsoft Windows\footnote{Microsof have relased a specification and promised not to change it so others also can use the .vhd format \cite{VHD_Wiki}} and is used to host a separate hard drive on the file system with features such as its own partitions, file system etc. but, the .vhd files 'live' in a file on the host operating system \cite{VHD_Wiki}. The .vhd format is used when hosting a virtual machine on a windows pc. Autopsy identified it as an NTFS file system, accessing it shows "un-allocated Blocks" and it seems to be encrypted (Autopsy identified that "vol 2" is as encrypted using bitlocker), see Appendix \ref{app:SecretDriveEncrypted}.


\subsubsection{Do any of the suspicious items discovered look suspicious?}
Selma Bouvier\footnote{One of Marge Simpsons chain smoking sisters} Has a link to some virtual hard discs (.vhd) on her desktop with "Super Secret" in the filename one of them flagged by Autopsy see Figure \ref{fig:SelmaBouvierSecredStuff}, the file is empty, the other file is the encrypted MoreSecretStuff.vhd previously found by Autopsy.

\begin{figure}[!ht] % Single column :figure	
	\centering
	\includegraphics[width=1\linewidth]{SelmaBouvierSecretStuff.png}
	\caption{The "Selma Bouvier" user has some interesting things on her desktop}
	\label{fig:SelmaBouvierSecredStuff}
\end{figure}


Poking some more Around Selma Bouvier's data revals a Readme.txt with a link to a Simpsons episode, where Homer gets hod of an "auto dialer" ST-5000. An .exe file named ST-5000 is also present in the profile's OneDrive.
\subsubsection{Other Suspicious Things}
I think the recent download (via chrome) and subsequent installation and execution of teamviewer in February and March of 2019 looks suspicious, see details in Appendix \ref{app:TeamViewer} TeamViewer is often used by scammers to take control of a victims computer.


\section{Conclusion}
The disc image was ingested and analyzed using Autopsy, Ingestion is a lengthy process depending on which Autopsy analysis modules are run on the image. In general the exercise gave a good tour of Autopsy.




\clearpage
\printbibliography % Output the bibliography   

%----------------------------------------------------------------------------------------



%----------------------------------------------------------------------------------------
%	 Appendices
%----------------------------------------------------------------------------------------
%
%\appendix


\clearpage
\chapter{Appendices}
\begin{appendices}

	\section{Image MetaData}\label{app:MetaData}


	\begin{figure}[!h] % Single column :figure	
		\centering
		\includegraphics[width=0.5\linewidth]{WindowsImageMetadata.png}
		\caption{Meta data for the Windows image.E01 Data source}
		\label{fig:windowsIageMetadata}
	\end{figure}


	\begin{figure}[!h] % Single column :figure	
		\centering
		\includegraphics[width=1\linewidth]{MoresecretStuffMetadata.png}
		\caption{Meta data for the "EvenMoreSecretStuff" Data source}
		\label{fig:evenMoreSecretStuffMetaData}
	\end{figure}


	\clearpage
	\section{January 1. 1980}\label{app:Jan1980}


	\begin{figure}[!h] % Single column :figure	
		\centering
		\includegraphics[width=1\linewidth]{Jan1980.png}
		\caption{Timeline results for January 1. 1980}
		\label{fig:Jan1980}
	\end{figure}

	\section{VHD file encrypted}\label{app:SecretDriveEncrypted}

	\begin{figure}[!h] % Single column :figure	
		\centering
		\includegraphics[width=1\linewidth]{EncryptionDetected.png}
		\caption{The "EvenMoreSecretStuff" virtual drive, seems encrypted using bitlocker}
		\label{fig:MoreSecretStuffEncrypted}
	\end{figure}

	\section{IP Hits}\label{app:IpHits}

	\begin{figure}[!h] % Single column :figure	
		\centering
		\includegraphics[width=1\linewidth]{ip-hits.png}
		\caption{Hits searching for ip v.4 addresses, the 6.0.0.0 hit is associated with the version of a windows subsystem and it not an IP in this context}
		\label{fig:IpHits}
	\end{figure}

	\section{TeamViewer}\label{app:TeamViewer}

	\begin{figure}[!h] % Single column :figure	
		\centering
		\includegraphics[width=1\linewidth]{TeamViewerDownload.png}
		\caption{TeamViewer was downloaded on January 25 2019 at 12:39pm}
		\label{fig:TeamViewerDownload}
	\end{figure}

	\begin{figure}[!h] % Single column :figure	
		\centering
		\includegraphics[width=1\linewidth]{TeamViewerInstallation.png}
		\caption{TeamViewer was installed on February 3rd 2019 at at 9:50pm}
		\label{fig:TeamViewerInstallation}
	\end{figure}


	\begin{figure}[!h] % Single column :figure	
		\centering
		\includegraphics[width=1\linewidth]{TeamViewerExe.png}
		\caption{TeamViewer was last run on February 20rd 2019 at at 1:57pm}
		\label{fig:TeamViewerexe}
	\end{figure}




\end{appendices}




\end{document}
